 \documentclass{article}
%include polycode.fmt
\usepackage[a4paper,left=2.5cm,right=2.5cm,top=\dimexpr15mm+1.5\baselineskip,bottom=2cm]{geometry}
\usepackage{amsmath}
\usepackage{graphicx}
\usepackage{setspace}
\usepackage{amsthm}
\usepackage{amssymb}
\usepackage{enumerate}
\usepackage{titlesec}
\usepackage{hyperref}
\usepackage{exscale}
\usepackage{pdfpages}
\usepackage{mathpartir}
\usepackage{relsize}
\usepackage{cases}
\usepackage{mathrsfs}
\usepackage{dutchcal}
\usepackage{float}
\usepackage{tcolorbox}
\usepackage{array}
\usepackage{xcolor}
\usepackage{stmaryrd}
\usepackage{calrsfs}
\usepackage{mdframed}
\usepackage{minted}
\DeclareMathAlphabet{\pazocal}{OMS}{zplm}{m}{n}

\theoremstyle{definition}
% \newtheorem{theorem}{Theorem}[section]
\newtheorem*{remark}{Remark}
% \newtheorem{corollary}{Corollary}[theorem]
\definecolor{bg}{rgb}{0.95,0.95,0.95}
\newcommand{\ip}[1]{\langle #1 \rangle}
\newcommand{\lp}{\cdot \mathtt{l}}
\newcommand{\rp}{\cdot \mathtt{r}}
\newcommand{\li}{\mathtt{l}\cdot}
\newcommand{\ri}{\mathtt{r}\cdot}
% \SetWatermarkText{\mdfivesum file {\jobname}}
% \SetWatermarkFontSize{1cm}
\allowdisplaybreaks
\graphicspath{ {./figures/} }

% Change here
\newcommand{\course}{SPELL}
\newcommand{\asnum}{2}
\newcommand{\true}{\,\textsf{true}}
\newcommand{\brs}[1]{\llbracket#1\rrbracket_\sigma}
\newcommand{\lb}{$\lambda$}
\newcommand{\case}[5]{\mathtt{case}\, #1\, \{\li #2\hookrightarrow #3\, |\, \ri #4\hookrightarrow #5\}}

\setcounter{section}{3}
\hypersetup{
    colorlinks, linkcolor=black, urlcolor=cyan
}
\title{\vspace{-3em}\course\, Lecture Notes \asnum\footnote{This note is licensed under a \href{https://creativecommons.org/licenses/by-nc-sa/2.0/}{CC BY-NC-SA 2.0} license.}}
\author{Y. Xiang\vspace{1em}}
\date{\today\vspace{-1em}}
\begin{document}
\maketitle

\subsection{Y combinator}
If we have defined the Y combinator:
\begin{align}
    \mathtt{Y} \triangleq \lambda f.(\lambda x.f (x x)) (\lambda x.f (x x))
\end{align}
then the factorial can be written down: (where $\mathtt{isZero},\mathtt{mul}, \mathtt{pred}$ are functions defined in the last section)
\begin{align}
    \mathtt{factProto} & \triangleq \lambda f.\lambda x.\mathtt{isZero}\, x\, 1\, (\mathtt{mul}\, x\, (f (\mathtt{pred}\, x))) \\
    \mathtt{fact}      & \triangleq \mathtt{Y}\, \mathtt{factProto}
\end{align}

\subsection{Type isomorphism}
\subsubsection*{Commutativity and associativity of sum and product type}
\begin{enumerate}
    \item Define functions:
          \begin{align*}
              f:A \mapsto B,\qquad f(e)\triangleq \ip{e\rp, e\lp} \\
              g:B \mapsto A,\qquad g(e)\triangleq \ip{e\rp, e\lp}
          \end{align*}

          For every $e: B\times A$,
          \begin{align*}
              f(g(e)) & = f(\ip{e\rp,e\lp})                         \\
                      & = \ip{\ip{e\rp,e\lp}\rp, \ip{e\rp,e\lp}\lp} \\
                      & = \ip{e\lp, e\rp}                           \\
                      & = e
          \end{align*}
          Similarly, for every $e: A\times B$, $g(f(e)) = e$.
    \item Define functions:
          \begin{align*}
              f:(A\times B)\times C \mapsto A\times (B\times C),\qquad f(e)\triangleq \ip{e\lp\lp,\ip{e\lp\rp,e\rp}} \\
              g:A\times (B\times C) \mapsto (A\times B)\times C,\qquad g(e)\triangleq \ip{\ip{e\lp, e\rp\lp},e\rp\rp}
          \end{align*}
          and easy to see that $f(g(e)) = e$ and $g(f(e)) = e$.
    \item Define functions:
          \begin{align*}
              f:(A+B)\mapsto (B+A),\qquad f(e)\triangleq \case{e}{x_1}{\ri x_1}{x_2}{\li x_2} \\
              g:(B+A)\mapsto (A+B),\qquad g(e)\triangleq \case{e}{x_1}{\ri x_1}{x_2}{\li x_2}
          \end{align*}
          and easy to see that $f(g(e)) = e$ and $g(f(e)) = e$.
    \item Define functions:
          \begin{align*}
              f    & :(A+B)+ C \mapsto A+(B+C)                                                                 \\
              f(e) & \triangleq \case{e}{x_1}{\case{x_1}{y_1}{\li y_1}{y_2}{\ri (\li y_2)}}{x_2}{\ri(\ri x_2)} \\
              g    & :A+(B+C) \mapsto (A+B)+ C                                                                 \\
              g(e) & \triangleq \case{e}{x_1}{\li(\li x_1)}{x_2}{\case{x_2}{y_1}{\li (\ri y_1)}{y_2}{\ri y_2}}
          \end{align*}
          and easy to see that $f(g(e)) = e$ and $g(f(e)) = e$.
\end{enumerate}
\subsubsection*{Refactoring}
\begin{enumerate}
    \item The following two codes are type-isomorphic:
          \begin{mdframed}[backgroundcolor=bg]
              \begin{minted}{elm}
type T = L1 T1 T2
       | L2 T1 T2
        \end{minted}
          \end{mdframed}
          $A:(T1\times T2) +(T1\times T2)$.
          \begin{mdframed}[backgroundcolor=bg]
              \begin{minted}{elm}
type T' = L1 T2
        | L2 T2
type alias T = (T1, T')
            \end{minted}
          \end{mdframed}
          $T':T2 + T2, B: T1 \times T' = T1 \times (T2 + T2)$.
          \begin{align*}
              f & : A\mapsto B,\qquad f(e) = \case{e}{x_1}{\ip{x_1\lp, \li (x_1\rp)}}{x_2}{\ip{x_2\lp, \ri (x_2\rp)}} \\
              g & : B\mapsto A,\qquad g(e) = \case{(e\rp)}{x_1}{\li \ip{e\lp, x_1}}{x_2}{\ri \ip{e\lp, x_2}}
          \end{align*}
    \item Functions can be curried or uncurried:
          \begin{mdframed}[backgroundcolor=bg]
              \begin{minted}{elm}
f1 (x, y) = e
f2 x y = e
  \end{minted}
          \end{mdframed}
          $F_1:A\times B \rightarrow \tau, F_2: A \rightarrow B \rightarrow \tau$.
          \begin{align*}
              f & : F_1 \mapsto F_2,\qquad f(e) = \lambda (x:A).\lambda (y:B).e\, \ip{x, y}  \\
              g & : F_2 \mapsto F_1,\qquad f(e) = \lambda (x: A\times B).e\, (x\lp)\, (x\rp)
          \end{align*}
    \item The following two ways of defining function are type-isomorphic:
          \begin{mdframed}[backgroundcolor=bg]
              \begin{minted}{elm}
f1 : T1 -> T
f2 : T2 -> T
        \end{minted}
          \end{mdframed}
          $A: (T1 \rightarrow T) \times (T2 \rightarrow T)$.
          \begin{mdframed}[backgroundcolor=bg]
              \begin{minted}{elm}
type T12 = L T1 | R T2
f : T12 -> T
        \end{minted}
          \end{mdframed}
          $B: (T1 + T2) \rightarrow T$.
          \begin{align*}
              f: A \mapsto B,\qquad f(e) = \lambda (x:(T1+T2)). \case{x}{x_1}{(e\lp)(x_1)}{x_2}{(e\rp)(x_2)}
          \end{align*}
\end{enumerate}

\begin{thebibliography}{9}
    \bibitem{int} Y. Yue, “The (Simply Typed) Lambda Calculus.” Mar. 12, 2023.
\end{thebibliography}

\end{document}
