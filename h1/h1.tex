\documentclass{article}
\usepackage[a4paper,left=2.5cm,right=2.5cm,top=\dimexpr15mm+1.5\baselineskip,bottom=2cm]{geometry}
\usepackage{amsmath}
\usepackage{graphicx}
\usepackage{setspace}
\usepackage{amsthm}
\usepackage{amssymb}
\usepackage{enumerate}
\usepackage{titlesec}
\usepackage{hyperref}
\usepackage{exscale}
\usepackage{pdfpages}
\usepackage{mathpartir}
\usepackage{relsize}
\usepackage{cases}
\usepackage{mathrsfs}
\usepackage{dutchcal}
\usepackage{float}
\usepackage{draftwatermark}
\usepackage{array}
\usepackage{xcolor}
\usepackage{stmaryrd}

\definecolor{dkgreen}{rgb}{0,0.6,0}
\definecolor{ltblue}{rgb}{0,0.4,0.4}
\definecolor{dkblue}{rgb}{0,0.1,0.5}
\definecolor{dkviolet}{rgb}{0.3,0,0.5}
\newcounter{rowcount}
\setcounter{rowcount}{0}

\theoremstyle{definition}
\newtheorem{theorem}{Theorem}[section]
\newtheorem*{remark}{Remark}
\newtheorem{corollary}{Corollary}[theorem]

% \SetWatermarkText{\mdfivesum file {\jobname}}
% \SetWatermarkFontSize{1cm}
\allowdisplaybreaks
\graphicspath{ {./figures/} }

% Change here
\newcommand{\course}{SPELL}
\newcommand{\asnum}{1}
\newcommand{\true}{\,\textsf{true}}
\newcommand{\brs}[1]{\llbracket#1\rrbracket_\sigma}
%

\setcounter{section}{1}
\hypersetup{
    colorlinks, linkcolor=black, urlcolor=cyan
}
\title{\vspace{-3em}\course\, Lecture Notes \asnum\footnote{This note is licensed under a \href{https://creativecommons.org/licenses/by-nc-sa/2.0/}{CC BY-NC-SA 2.0} license.}}
\author{Y. Xiang\vspace{1em}}
\date{\today\vspace{-1em}}
\begin{document}
\maketitle
% \begin{spacing}{2}
\subsection{Intuitionistic Proofs}

I used some linear-style proofs for most theorems to represent the proof tree which are much \emph{harder} to draw and read on an A-4 paper. They can be converted to each other easily.

\begin{theorem}
    $\vdash P\supset P\true$.
    \label{imp-ii}
\end{theorem}
\setcounter{rowcount}{0}
\begin{proof}
    \begin{tabular}{@{\stepcounter{rowcount}\therowcount. }lr}
        $P\true\vdash P\true$,    & \textsc{Assump} \\
        $\vdash P\supset P\true$, & $\supset$-I     \\
    \end{tabular}
\end{proof}

\setcounter{rowcount}{0}
\begin{theorem}
    $\vdash P \supset ((P\supset Q)\supset Q)$ \true.
\end{theorem}
\begin{proof}
    \begin{tabular}{@{\stepcounter{rowcount}\therowcount. }lr}
        $P,P\supset Q \true \vdash P\true$,                           & \textsc{Assump} \\
        $P,P\supset Q \true \vdash P\supset Q\true$,                  & \textsc{Assump} \\
        $P,P\supset Q \true \vdash Q \true$,                          & 1,2,$\supset$-E \\
        $P\true \vdash (P\supset Q) \supset Q \true$                  & 3,$\supset$-I   \\
        $\vdash P \supset \left( (P\supset Q) \supset Q \right)\true$ & 4,$\supset$-I   \\
    \end{tabular}
\end{proof}

Since we define negation $\neg p$ by $p \supset \bot$, and we have the following corollary:
\begin{corollary}
    \textbf{(Double Negation Introduction, $\neg \neg$-I)}
    $\vdash P\supset \neg \neg P$\true.
\end{corollary}

However, the \textbf{double negation elimination}, $\neg \neg P \supset P$ is \emph{not} an intuitionistic tautology,
which means that $\neg \neg P$ is weaker than $P$.

For simplicity, we add a rule which is equivalent to $\supset$-I but written in language of $\neg$:

\noindent \textbf{Negation Introduction, $\neg$-I}
\begin{mathpar}
    \inferrule{\Gamma, P\true\vdash Q\true\\ \Gamma, P\true \vdash \neg Q\true}{\Gamma \vdash \neg P\true}
\end{mathpar}

\setcounter{rowcount}{0}
\begin{theorem}
    \textbf{(Embedding of LEM)}
    $\vdash \neg \neg (P \lor \neg P)$ \true.
    \label{embedding-lem}
\end{theorem}
\begin{proof}
    \begin{tabular}{@{\stepcounter{rowcount}\therowcount. }lr}
        $\neg(P\lor \neg P), P \true \vdash P\lor \neg P \true$,       & $\lor$-Il       \\
        $\neg(P\lor \neg P), P \true \vdash \neg(P\lor \neg P) \true$, & \textsc{Assump} \\
        $\neg(P\lor \neg P)\true \vdash \neg P\true$,                  & 1,2,$\neg$-I    \\
        $\neg P\true \vdash P\lor \neg P \true$,                       & $\lor$-Ir       \\
        $\vdash \neg P \supset (P\lor \neg P) \true$,                  & 4,$\supset$-I   \\
        $\neg(P\lor \neg P)\true \vdash P\lor \neg P \true$,           & 3,5,$\supset$-E \\
        $\neg(P\lor \neg P)\true \vdash \neg(P\lor \neg P) \true$,     & \textsc{Assump} \\
        $\vdash \neg \neg(P\lor \neg P) \true$,                        & $\neg$-I        \\
    \end{tabular}
\end{proof}

\setcounter{rowcount}{0}
\begin{theorem}
    \textbf{(Triple Negation Reduction)}
    $\vdash \neg \neg \neg P \supset \neg P$ \true.
    \label{tnr}
\end{theorem}
\begin{proof}
    \begin{tabular}{@{\stepcounter{rowcount}\therowcount. }lr}
        $\neg \neg \neg P, P\true \vdash \neg \neg P\true$,      & $\neg \neg$-I   \\
        $\neg \neg \neg P, P\true \vdash \neg \neg \neg P\true$, & \textsc{Assump} \\
        $\neg \neg \neg P\true \vdash \neg P\true$,              & 1,2,$\neg$-I    \\
        $\vdash \neg \neg \neg P \supset \neg P\true$,           & 3,$\supset$-I   \\
    \end{tabular}
\end{proof}

Theorem \ref{tnr} implies an important fact:
\begin{theorem}
    $\Gamma \vdash_{IPL} \neg P \true$ \emph{iff} $\Gamma \vdash_{CPL} \neg P \true$.
\end{theorem}
\begin{proof}
    Here we regard CPL as an extension to IPL by adding LEM.
    Since $P\supset \neg\neg P\true$ in IPL and theorem \ref{embedding-lem}, we know that:
    \[
        \Gamma \vdash_{IPL} \neg\neg P\true \, \text{if}\, \Gamma \vdash_{CPL} P\true
    \]
    From theorem \ref{tnr},
    \[
        \Gamma \vdash_{IPL} \neg P\true \, \text{if}\, \Gamma \vdash_{CPL} \neg P\true
    \]
\end{proof}

\begin{theorem}
    $\vdash \neg\neg (\neg\neg P\supset P)\true$.
\end{theorem}
\begin{proof}
    To save the page, I omit all the ``\textsf{true}'' notations in the following proof tree.
    \begin{mathpar}
        \inferrule*[Right=$\neg$-I]{
            \inferrule*[Right=$\supset$-I]{
                \inferrule*[Right=$\bot$-E]{
                    \inferrule*[Right=$\supset$-E]{
                        \cdots \\
                        \inferrule*[Right=Ass.]{ }{\neg (\neg\neg P \supset P), \neg\neg P\vdash \neg\neg P}
                    }{\neg (\neg\neg P \supset P), \neg\neg P\vdash \bot}}{
                    \neg (\neg\neg P \supset P), \neg\neg P\vdash P
                }}{
                \neg (\neg\neg P \supset P)\vdash \neg\neg P\supset P
            }\\
            \inferrule*[Right=Ass.]{ }{\neg (\neg\neg P \supset P)\vdash \neg (\neg\neg P \supset P)}
        }{
            \vdash \neg\neg (\neg\neg P \supset P)
        }
    \end{mathpar}
    where $\cdots$ continues here:
    \begin{mathpar}

        \inferrule*[Right=$\neg$-I,leftskip=5em]{\inferrule*[Right=$\supset$-I]{\inferrule*[Right=Ass.]{ }{\neg (\neg\neg P \supset P), \neg\neg P, P\vdash P}}{
                \neg (\neg\neg P \supset P), \neg\neg P, P\vdash \neg\neg P \supset P
            }\\
            \inferrule*[Right=Ass.]{ }{\neg (\neg\neg P \supset P), \neg\neg P, P\vdash \neg (\neg\neg P \supset P)}
        }{\neg (\neg\neg P \supset P), \neg\neg P\vdash \neg P}
    \end{mathpar}
\end{proof}

This is the double negation embedding of the double negation elimination. Since the double negation embedding will \emph{weaken} the judgement, this theorem doesn't mean ``IPL refutes double negation elimination''.

\subsection{Classical Logic from Intuitionistic Logic}

There are three ways to recover the classical logic from intuitionistic logic, by admitting any one of three axioms.

\noindent\textbf{Double Negation Elimination (DNE)}
\begin{mathpar}
    \inferrule*[Right=DNE]{ }{\Gamma \vdash \neg\neg P \supset P\true}
\end{mathpar}

\noindent\textbf{Law of Excluded Middle (LEM)}
\begin{mathpar}
    \inferrule*[Right=LEM]{ }{\Gamma \vdash P\lor \neg P \true}
\end{mathpar}

\noindent\textbf{A3}
\begin{mathpar}
    \inferrule*[Right=A3]{ }{\Gamma \vdash (\neg P \supset \neg Q)\supset (\neg P \supset Q)\supset P \true}
\end{mathpar}

\begin{remark}
    LEM, DNE and A3 are not judgements or propositions. They are called \emph{rules}.
\end{remark}

Let's prove their equivalency by using circular ``implication'': DNE implies LEM implies A3 implies DNE.
Here ``$A$ implies $B$'' means $\Gamma \vdash_{IPL\cup \{A\}} P\true$ iff $\Gamma \vdash_{IPL\cup \{B\}} P\true$. $\vdash_{IPL}$ means that the derivation is under IPL rules.

\begin{theorem}
    DNE implies rule LEM.
\end{theorem}
\begin{proof}
    Theorem \ref{embedding-lem} with DNE imply LEM.
\end{proof}

\begin{theorem}
    LEM implies rule A3.
\end{theorem}
\setcounter{rowcount}{0}
\begin{proof}
    Let $\Gamma' = \Gamma \cup\{\neg P, (\neg P \supset \neg Q), (\neg P \supset Q)\}$.

    \begin{tabular}{@{\stepcounter{rowcount}\therowcount. }lr}
        $\Gamma, P, (\neg P \supset \neg Q), (\neg P \supset Q)\true \vdash P \true $,                & \textsc{Assump}     \\
        $\Gamma'\true \vdash \neg P\true$,                                                            & \textsc{Assump}     \\
        $\Gamma'\true \vdash \neg P \supset \neg Q\true$,                                             & \textsc{Assump}     \\
        $\Gamma'\true \vdash \neg P \supset Q\true$,                                                  & \textsc{Assump}     \\
        $\Gamma'\true \vdash Q\true$,                                                                 & 2,4,$\supset$-E     \\
        $\Gamma'\true \vdash \neg Q\true$,                                                            & 2,3,$\supset$-E     \\
        $\Gamma'\true \vdash \bot\true$,                                                              & 5,6,$\supset$-E     \\
        $\Gamma'\true \vdash P\true$,                                                                 & $\bot$-E            \\
        $\Gamma, \neg P\true \vdash (\neg P \supset \neg Q)\supset (\neg P \supset Q)\supset P\true$, & 8,$\supset$-I twice \\
        $\Gamma, P\true \vdash (\neg P \supset \neg Q)\supset (\neg P \supset Q)\supset P\true$,      & 1,$\supset$-I twice \\
        $\Gamma \true \vdash P\lor \neg P$,                                                           & LEM                 \\
        $\Gamma \true \vdash (\neg P \supset \neg Q)\supset (\neg P \supset Q)\supset P\true$,        & 9,10,11,$\lor$-E    \\
    \end{tabular}
\end{proof}

\setcounter{rowcount}{0}
\begin{theorem}
    A3 implies rule DNE.
\end{theorem}
\begin{proof}
    Let $\Gamma' = \Gamma \cup \{\neg\neg P\}$.

    \begin{tabular}{@{\stepcounter{rowcount}\therowcount. }lr}
        $\Gamma' \true \vdash \neg P \supset \neg P\true$,                                        & Theorem \ref{imp-ii} \\
        $\Gamma' \true \vdash (\neg P \supset \neg P) \supset (\neg P \supset P)\supset P\true $, & A3                   \\
        $\Gamma' \true \vdash (\neg P \supset P)\supset P\true $,                                 & 1,2,$\supset$-E      \\
        $\Gamma' \true \vdash \neg P\supset P \true$                                              & \textsc{Assump}      \\
        $\Gamma' \true \vdash P \true$                                                            & 3,4,$\supset$-E      \\
        $\Gamma \true \vdash \neg\neg P \supset P \true$                                          & 5,$\supset$-I        \\
    \end{tabular}
\end{proof}

Hence, we could say the DNE, LEM and A3 are equivalent extension to IPL.

\subsection{The Main Theorem}

In \cite{int}, we want to prove:

\begin{theorem}
    If $\Gamma \vdash_{I} P \true$ then $\Gamma \vdash_{C} P \true$.
\end{theorem}

and we introduced the method of \emph{truth table} to transform the theorem into:

\[
    \Gamma \vdash_{I} P \true \Rightarrow \forall_\sigma, \brs{\Gamma} \leq \brs{P}.
\]

We use the \emph{structural induction} on the derivation tree and we only need to prove the theorem on the bottom most rules.

For example,

\begin{theorem}
    (\textbf{Assump}) $\forall_\sigma, \brs{\Gamma, A} \leq \brs{A}$.
\end{theorem}

\begin{proof}
    By definition, it is equivalent to $\brs{\Gamma} \times \brs{A} \leq \brs{A}$.
    If $\brs{A} = 1$, then $\brs{\Gamma}\leq 1 = \brs{A}$;
    If $\brs{A} = 0$, then $0\leq 0$.
\end{proof}

Proofs for other rules are quite tedious and trivial.

\subsection*{Appendix}
See the attached file \textsf{h1.v}.

\begin{thebibliography}{9}
    \bibitem{int} Y. Yue and H. R., “Intuitionistic Propositional Logic and its validity in Classical Logic.” Feb. 27, 2023.
\end{thebibliography}

\end{document}
