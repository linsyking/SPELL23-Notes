\documentclass{article}
%include polycode.fmt
\usepackage[a4paper,left=2.5cm,right=2.5cm,top=\dimexpr15mm+1.5\baselineskip,bottom=2cm]{geometry}
\usepackage{amsmath}
\usepackage{graphicx}
\usepackage{setspace}
\usepackage{amsthm}
\usepackage{amssymb}
\usepackage{enumerate}
\usepackage{titlesec}
\usepackage{hyperref}
\usepackage{exscale}
\usepackage{pdfpages}
\usepackage{mathpartir}
\usepackage{relsize}
\usepackage{cases}
\usepackage{mathrsfs}
\usepackage{dutchcal}
\usepackage{float}
\usepackage{tcolorbox}
\usepackage{array}
\usepackage{xcolor}
\usepackage{stmaryrd}
\usepackage{calrsfs}
\usepackage{mdframed}
\usepackage{minted}
\DeclareMathAlphabet{\pazocal}{OMS}{zplm}{m}{n}

\theoremstyle{definition}
% \newtheorem{theorem}{Theorem}[section]
\newtheorem*{lemma}{Lemma}
\newtheorem*{remark}{Remark}
% \newtheorem{corollary}{Corollary}[theorem]
\definecolor{bg}{rgb}{0.95,0.95,0.95}
\newcommand{\ip}[1]{\langle #1 \rangle}
\newcommand{\lp}{\cdot \mathtt{l}}
\newcommand{\rp}{\cdot \mathtt{r}}
\newcommand{\li}{\mathtt{l}\cdot}
\newcommand{\ri}{\mathtt{r}\cdot}
% \SetWatermarkText{\mdfivesum file {\jobname}}
% \SetWatermarkFontSize{1cm}
\allowdisplaybreaks
\graphicspath{ {./figures/} }

% Change here
\newcommand{\course}{SPELL}
\newcommand{\asnum}{4}
\newcommand{\true}{\,\textsf{true}}
\newcommand{\brs}[1]{\llbracket#1\rrbracket_\sigma}
\newcommand{\lb}{$\lambda$}
\newcommand{\case}[5]{\mathtt{case}\, #1\, \{\li #2\hookrightarrow #3\, |\, \ri #4\hookrightarrow #5\}}

\setcounter{section}{\asnum}
\hypersetup{
    colorlinks, linkcolor=black, urlcolor=cyan
}
\title{\vspace{-3em}\course\, Lecture Notes \asnum\footnote{This note is licensed under a \href{https://creativecommons.org/licenses/by-nc-sa/2.0/}{CC BY-NC-SA 2.0} license.}}
\author{Y. Xiang\vspace{1em}}
\date{\today\vspace{-1em}}
\begin{document}
\maketitle

\subsection{Properties}

\subsubsection*{Substitution rule}

The subsection lemma says:
\begin{lemma}
    If $\Gamma \vdash e':\tau'$ and $\Gamma, x:\tau'\vdash e:\tau$, then $\Gamma \vdash [e'/x]e:\tau$.
\end{lemma}

We use the rule induction on ``$\Gamma, x:\tau'\vdash e:\tau$'' to prove the lemma.

Here we only prove the lemma case for $\mathtt{plus}(e_1;e_2)$.
We pattern match ``$\Gamma, x:\tau'\vdash e:\tau$'' to the \textsc{TPlus} rule:
\begin{mathpar}
    \inferrule*{\Gamma, x:\tau'\vdash e_1:\mathtt{num}  \\  \Gamma, x:\tau'\vdash e_2:\mathtt{num} }{ \Gamma, x:\tau'\vdash \mathtt{plus}(e_1;e_2): \mathtt{num} }
\end{mathpar}

By definition, $[e'/x]\mathtt{plus}(e_1,e_2) = \mathtt{plus}([e'/x]e_1, [e'/x]e_2)$. Applying IH, we have $\Gamma \vdash [e'/x]e_1:\mathtt{num}$ and $\Gamma \vdash [e'/x]e_2:\mathtt{num}$. Then by the \textsc{TPlus} rule, we have $\Gamma \vdash \mathtt{plus}([e'/x]e_1, [e'/x]e_2):\mathtt{num}$, which is what we want.

\subsubsection*{Unicity of typing}

The unicity lemma says:
\begin{lemma}
    For all $\Gamma, e$, if $\Gamma \vdash e: \tau$ and $\Gamma \vdash e: \tau'$, then $\tau = \tau'$.
\end{lemma}

\begin{proof}
    We do rule induction.
    \begin{enumerate}[i)]
        \item Case $\inferrule*[Right=TVar]{ }{\Gamma \vdash e : \tau}$

              Here $e$ is a free variable, then we can read its type directly, which is the ``variable'' type.
        \item Case $\inferrule*[Right=TNum]{ }{\Gamma \vdash \mathtt{num}\ip{n} : \mathtt{num}}$

              Here $\mathtt{num}\ip{n}$ is a number, then we can read its type directly, which is the ``numbers'' type.
        \item Case $\inferrule*[Right=TStr]{ }{\Gamma \vdash \mathtt{str}\ip{n} : \mathtt{str}}$

              Here $\mathtt{str}\ip{n}$ is a number, then we can read its type directly, which is the ``strings'' type.
        \item Case $\inferrule*[Right=TLen]{\Gamma \vdash e':\mathtt{str}}{\Gamma \vdash \mathtt{len}(e') : \mathtt{num}}$

              We can also read the type directly from $\mathtt{len}(e')$, which is the ``numbers'' type.
        \item Case $\inferrule*[Right=TPlus]{\Gamma \vdash e_1:\mathtt{num}\\ \Gamma \vdash e_2:\mathtt{num}}{\Gamma \vdash \mathtt{plus}(e_1;e_2) : \mathtt{num}}$

              We can also read the type directly from $\mathtt{plus}(e_1;e_2)$, which is the ``numbers'' type.
        \item Case $\inferrule*[Right=TCat]{\Gamma \vdash e_1:\mathtt{str}\\ \Gamma \vdash e_2:\mathtt{str}}{\Gamma \vdash \mathtt{cat}(e_1;e_2) : \mathtt{str}}$

              We can also read the type directly from $\mathtt{cat}(e_1;e_2)$, which is the ``strings'' type.
        \item Case $\inferrule*[Right=TLet]{\Gamma \vdash e_1:\tau_1 \\ \Gamma,x:\tau_1 \vdash e_2:\tau_2}{\Gamma \vdash \mathtt{let}(e_1;x.e_2) : \tau_2}$

              Suppose we have $\inferrule*{\Gamma \vdash e_1:\tau_1' \\ \Gamma,x:\tau_1' \vdash e_2:\tau_2'}{\Gamma \vdash \mathtt{let}(e_1;x.e_2) : \tau_2'}$.

              Applying IH on $\Gamma \vdash e_1:\tau_1$, we know that $e_1$ is uniquely typed, i.e. $\tau_1 = \tau_1'$.
              Applying IH on $\Gamma,x:\tau_1 \vdash e_2:\tau_2$, we know that $e_2$ is also uniquely typed, i.e. $\tau_2 = \tau_2'$.
    \end{enumerate}
\end{proof}

\begin{thebibliography}{9}
    \bibitem{int} Y. Yue. J. YuChen, “Proving Metatheorems and Type Safety.” Mar. 12, 2023.
\end{thebibliography}

\end{document}
