\documentclass{article}
%include polycode.fmt
\usepackage[a4paper,left=2.5cm,right=2.5cm,top=\dimexpr15mm+1.5\baselineskip,bottom=2cm]{geometry}
\usepackage{amsmath}
\usepackage{graphicx}
\usepackage{setspace}
\usepackage{amsthm}
\usepackage{amssymb}
\usepackage{enumerate}
\usepackage{titlesec}
\usepackage{hyperref}
\usepackage{exscale}
\usepackage{pdfpages}
\usepackage{mathpartir}
\usepackage{relsize}
\usepackage{cases}
\usepackage{mathrsfs}
\usepackage{dutchcal}
\usepackage{float}
\usepackage{tcolorbox}
\usepackage{array}
\usepackage{xcolor}
\usepackage{stmaryrd}
\usepackage{calrsfs}
\usepackage{mdframed}
\usepackage{minted}
\DeclareMathAlphabet{\pazocal}{OMS}{zplm}{m}{n}

\theoremstyle{definition}
\newtheorem{theorem}{Theorem}[section]
\newtheorem{lemma}{Lemma}[section]
\newtheorem*{remark}{Remark}
\newtheorem{corollary}{Corollary}[section]
\definecolor{bg}{rgb}{0.95,0.95,0.95}
\newcommand{\ip}[1]{\langle #1 \rangle}
\newcommand{\lp}{\cdot \mathtt{l}}
\newcommand{\rp}{\cdot \mathtt{r}}
\newcommand{\li}{\mathtt{l}\cdot}
\newcommand{\ri}{\mathtt{r}\cdot}
% \SetWatermarkText{\mdfivesum file {\jobname}}
% \SetWatermarkFontSize{1cm}
\allowdisplaybreaks
\graphicspath{ {./figures/} }

% Change here
\newcommand{\course}{SPELL}
\newcommand{\asnum}{4}
\newcommand{\true}{\,\textsf{true}}
\newcommand{\brs}[1]{\llbracket#1\rrbracket_\sigma}
\newcommand{\lb}{$\lambda$}
\newcommand{\case}[5]{\mathtt{case}\, #1\, \{\li #2\hookrightarrow #3\, |\, \ri #4\hookrightarrow #5\}}

\setcounter{section}{\asnum}
\hypersetup{
    colorlinks, linkcolor=black, urlcolor=cyan
}
\title{\vspace{-3em}\course\, Lecture Notes \asnum\footnote{This note is licensed under a \href{https://creativecommons.org/licenses/by-nc-sa/2.0/}{CC BY-NC-SA 2.0} license.}\quad}
\author{Y. Xiang\vspace{1em}}
\date{\today\vspace{-1em}}
\begin{document}
\maketitle

\subsection{Properties}

\subsubsection*{Substitution rule}

The subsection lemma says:
\begin{lemma}
    \label{lem:substitution}
    If $\Gamma \vdash e':\tau'$ and $\Gamma, x:\tau'\vdash e:\tau$, then $\Gamma \vdash [e'/x]e:\tau$.
\end{lemma}

\begin{proof}
    We use the rule induction on ``$\Gamma, x:\tau'\vdash e:\tau$'' to prove the lemma.

    Here we only prove the lemma case for $\mathtt{plus}(e_1;e_2)$.
    We pattern match ``$\Gamma, x:\tau'\vdash e:\tau$'' to the \textsc{TPlus} rule:
    \begin{mathpar}
        \inferrule*{\Gamma, x:\tau'\vdash e_1:\mathtt{num}  \\  \Gamma, x:\tau'\vdash e_2:\mathtt{num} }{ \Gamma, x:\tau'\vdash \mathtt{plus}(e_1;e_2): \mathtt{num} }
    \end{mathpar}

    By definition, $[e'/x]\mathtt{plus}(e_1,e_2) = \mathtt{plus}([e'/x]e_1, [e'/x]e_2)$. Applying IH, we have $\Gamma \vdash [e'/x]e_1:\mathtt{num}$ and $\Gamma \vdash [e'/x]e_2:\mathtt{num}$. Then by the \textsc{TPlus} rule, we have $\Gamma \vdash \mathtt{plus}([e'/x]e_1, [e'/x]e_2):\mathtt{num}$, which is what we want.
\end{proof}

\subsubsection*{Unicity of typing}

The unicity lemma says:
\begin{lemma}
    For all $\Gamma, e$, if $\Gamma \vdash e: \tau$ and $\Gamma \vdash e: \tau'$, then $\tau = \tau'$.
\end{lemma}

\begin{proof}
    We do rule induction.
    \begin{itemize}
        \item Case $\inferrule*[Right=TVar]{ }{\Gamma \vdash e : \tau}$

              Here $e$ is a free variable, then we can read its type directly, which is the ``variable'' type.
        \item Case $\inferrule*[Right=TNum]{ }{\Gamma \vdash \mathtt{num}\ip{n} : \mathtt{num}}$

              Here $\mathtt{num}\ip{n}$ is a number, then we can read its type directly, which is the ``numbers'' type.
        \item Case $\inferrule*[Right=TStr]{ }{\Gamma \vdash \mathtt{str}\ip{n} : \mathtt{str}}$

              Here $\mathtt{str}\ip{n}$ is a number, then we can read its type directly, which is the ``strings'' type.
        \item Case $\inferrule*[Right=TLen]{\Gamma \vdash e':\mathtt{str}}{\Gamma \vdash \mathtt{len}(e') : \mathtt{num}}$

              We can also read the type directly from $\mathtt{len}(e')$, which is the ``numbers'' type.
        \item Case $\inferrule*[Right=TPlus]{\Gamma \vdash e_1:\mathtt{num}\\ \Gamma \vdash e_2:\mathtt{num}}{\Gamma \vdash \mathtt{plus}(e_1;e_2) : \mathtt{num}}$

              We can also read the type directly from $\mathtt{plus}(e_1;e_2)$, which is the ``numbers'' type.
        \item Case $\inferrule*[Right=TCat]{\Gamma \vdash e_1:\mathtt{str}\\ \Gamma \vdash e_2:\mathtt{str}}{\Gamma \vdash \mathtt{cat}(e_1;e_2) : \mathtt{str}}$

              We can also read the type directly from $\mathtt{cat}(e_1;e_2)$, which is the ``strings'' type.
        \item Case $\inferrule*[Right=TLet]{\Gamma \vdash e_1:\tau_1 \\ \Gamma,x:\tau_1 \vdash e_2:\tau_2}{\Gamma \vdash \mathtt{let}(e_1;x.e_2) : \tau_2}$

              Suppose we have $\inferrule*{\Gamma \vdash e_1:\tau_1' \\ \Gamma,x:\tau_1' \vdash e_2:\tau_2'}{\Gamma \vdash \mathtt{let}(e_1;x.e_2) : \tau_2'}$.

              Applying IH on $\Gamma \vdash e_1:\tau_1$, we know that $e_1$ is uniquely typed, i.e. $\tau_1 = \tau_1'$.
              Applying IH on $\Gamma,x:\tau_1 \vdash e_2:\tau_2$, we know that $e_2$ is also uniquely typed, i.e. $\tau_2 = \tau_2'$.
    \end{itemize}
\end{proof}

\begin{remark}
    The simply typed lambda calculus also follows the unicity of typing. This is because there is no ambiguity in the typing rules and syntax.
\end{remark}

\subsubsection*{Uniquely deterministic}

Before we enter the topic, we first prove a lemma:

\begin{lemma}
    \label{lem:valorstep}
    If $e$ \textsf{val}, then there doesn't exist $e'$ such that $e\longrightarrow e'$.
\end{lemma}

\begin{proof}
    By syntax there are only two possibilities:
    \begin{itemize}
        \item Case $\inferrule*[Right=ValNum]{ }{\mathtt{num}\ip{n}\, \mathsf{val}}$

              Cannot find rule implying $\inferrule*[Right=???]{\cdots}{\mathtt{num}\ip{n}\longrightarrow e'}$

        \item Case $\inferrule*[Right=ValStr]{ }{\mathtt{str}\ip{s}\, \mathsf{val}}$

              Cannot find rule implying $\inferrule*[Right=???]{\cdots}{\mathtt{str}\ip{n}\longrightarrow e'}$

    \end{itemize}
    In both cases, we cannot find a rule that can step $e$.
    In other words, such $e'$ doesn't exist.
\end{proof}

\begin{corollary}
    \label{cor:uniquedet}
    If there exists $e'$ such that $e\longrightarrow e'$, then $e$ is not \textsf{val}.
\end{corollary}

The uniquely deterministic lemma says:
\begin{lemma}
    If $e\longrightarrow e'$, then $e'$ is unique.
\end{lemma}

\begin{proof}
    Here we only prove the cases where $e$ is a \texttt{let} or \texttt{cat}.
    Use rule induction on ``$e\longrightarrow e'$''.

    \begin{itemize}
        \item Case \texttt{let}

              In this case by syntax we have two possible derivations:
              \begin{itemize}
                  \item $\inferrule*[Right=DLet]{e_1 \, \mathsf{val}}{\mathtt{let}(e_1;x.e_2) \longrightarrow [e_1/x]e_2}$
                  \item $\inferrule*[Right=DLetL]{e_1\longrightarrow e_1'}{\mathtt{let}(e_1;x.e_2) \longrightarrow \mathtt{let}(e_1';x.e_2)}$
              \end{itemize}
              By lemma \ref{lem:valorstep}, if $e_1$ \textsf{val}, then the premise of \textsc{DLetL} cannot be satisfied, and we can only apply \textsc{DLet} rule. Vice versa, if $e_1\longrightarrow e_1'$, for some $e_1'$ then the premise of \textsc{DLet} cannot be satisfied, and we can only apply \textsc{DLetL} rule.

              Hence the result $[e_1/x]e_2$ is uniquely determined.
        \item Case \texttt{cat}

              By syntax we have three possible derivations:
              \begin{itemize}
                  \item $\inferrule*[Right=DCat]{ }{\mathtt{cat}(\mathtt{str}\ip{s_1};\mathtt{str}\ip{s_1})\longrightarrow \mathtt{str}\ip{s_1s_2}}$

                  \item $\inferrule*[Right=DCatL]{e_1\longrightarrow e_1'}{\mathtt{cat}(e_1;e_2)\longrightarrow \mathtt{cat}(e_1';e_2)}$
                  \item $\inferrule*[Right=DCatR]{e_1\,\mathsf{val} \\ e_2\longrightarrow e_2'}{\mathtt{cat}(e_1;e_2)\longrightarrow \mathtt{cat}(e_1;e_2')}$
              \end{itemize}
              Hence we can assume that $e = \mathtt{cat}(e_1;e_2)$ (there are no other syntax possible).
              By \textsc{ValStr}, the \textsc{DCat} case implies $e_1$ \textsf{val} and $e_2$ \textsf{val}.
              Then we apply the progress theorem on $e_1$ and $e_2$, we have following possibilities:
              \begin{itemize}
                  \item $e_1$ \textsf{val} and $e_2$ \textsf{val}.

                        By lemma \ref{lem:valorstep}, it contradicts \textsc{DCatL} and \textsc{DCatR}, hence only \textsc{DCat} rule is possible.
                  \item $e_1$ \textsf{val} and there exists $e_2'$ such that $e_2\longrightarrow e_2'$.

                        By lemma \ref{lem:valorstep}, it contradicts \textsc{DCatL} and \textsc{DCat}, hence only \textsc{DCatR} rule is possible.
                  \item There exists $e_1'$ such that $e_1\longrightarrow e_1'$ and $e_2$ \textsf{val}.

                        By lemma \ref{lem:valorstep}, it contradicts \textsc{DCatR} and \textsc{DCat}, hence only \textsc{DCatL} rule is possible.
                  \item There exists $e_1',e_2'$ such that $e_1\longrightarrow e_1'$ and $e_2\longrightarrow e_2'$.

                        By lemma \ref{lem:valorstep}, it contradicts \textsc{DCat} and \textsc{DCatR}, hence only \textsc{DCatL} rule are possible.
              \end{itemize}
    \end{itemize}
\end{proof}

\subsection{Type safety}

Type safety theorem includes two parts: progress and preservation.

\subsubsection*{Progress}

\begin{theorem}
    If $\vdash e:\tau$, then either $e$ \textsf{val} or there exists $e'$ such that $e\longrightarrow e'$.
\end{theorem}

\begin{proof}
    Here we prove the cases for \texttt{let} and \texttt{len}. (Other cases are on the \cite{int})
    \begin{itemize}
        \item Case $\inferrule*[Right=TLen]{\Gamma \vdash e':\mathtt{str}}{\Gamma \vdash \mathtt{len}(e') : \mathtt{num}}$

              WTS: Either $\mathtt{len}(e')\, \mathsf{val}$, or there exists $e''$ such that $\mathtt{len}(e') \longrightarrow \mathtt{len}(e'')$.

              From syntax we know $\Gamma$ is empty; from premise we know that $\vdash e': \mathtt{str}$. Therefore, IH claims that:
              \paragraph*{IH.} Either $e'$ \textsf{val} or there exists $e''$ such that $e' \longrightarrow e''$.

              Case on \textbf{IH}:
              \begin{itemize}
                  \item Case: $e'\, \mathsf{val}$

                        From \emph{Canonical Forms Lemma} (CFL) for type \texttt{str}, we know that $e' = \mathtt{str}\ip{s}$ for some $s$.
                        Therefore $\mathtt{len}(e') = \mathtt{len}(\mathtt{str}\ip{s}) \longrightarrow \mathtt{num}\ip{ |s| }$ by \textsc{DLen}.
                  \item Case: There exists $e''$ such that $e' \longrightarrow e''$

                        By \textsc{DLenL}, we have $\mathtt{len}(e')\longrightarrow \mathtt{len}(e'')$.
              \end{itemize}
        \item Case $\inferrule*[Right=TLet]{\Gamma \vdash e_1:\tau_1 \\ \Gamma,x:\tau_1 \vdash e_2:\tau_2}{\Gamma \vdash \mathtt{let}(e_1;x.e_2) : \tau_2}$

              WTS: Either $\mathtt{let}(e_1;x.e_2)\, \mathsf{val}$, or there exists $e''$ such that $\mathtt{let}(e_1;x.e_2) \longrightarrow e''$.

              From syntax we know $\Gamma$ is empty; from premise we know that $\vdash e_1:\tau_1$ and $x:\tau_1 \vdash e_2:\tau_2$. Therefore,
              \paragraph*{IH.} Either $e_1$\, \textsf{val} or there exists $e_1''$ such that $e_1 \longrightarrow e_1''$.

              Case on \textbf{IH}:
              \begin{itemize}
                  \item Case: $e_1\, \mathsf{val}$

                        By \textsc{DLet}, we have $\mathtt{let}(e_1;x.e_2) \longrightarrow [e_1/x]e_2$.

                  \item Case: There exists $e_1''$ such that $e_1 \longrightarrow e_1''$

                        By \textsc{DLetL}, we have $\mathtt{let}(e_1;x.e_2) \longrightarrow \mathtt{let}(e_1'';x.e_2)$.
              \end{itemize}
    \end{itemize}
\end{proof}

\subsubsection*{Preservation}

\begin{theorem}
    If $\vdash e:\tau$ and $e\longrightarrow e'$, then $\vdash e':\tau$.
\end{theorem}

\begin{proof}
    It's possible to do induction on the derivation of $\vdash e:\tau$, but it's much harder (\cite{int}). Instead, we can do induction on the derivation of $e\longrightarrow e'$. Here we only show the case where $e$ is \texttt{let}.

    \begin{itemize}
        \item Case $\inferrule*[Right=DLet]{e_1\, \mathsf{val}}{\mathtt{let}(e_1;x.e_2) \longrightarrow [e_1/x]e_2}$.

              By pattern match we know that $e = \mathtt{let}(e_1;x.e_2), e' = [e_1/x]e_2$

              WTS: If $\vdash e:\tau$, then $\vdash [e_1/x]e_2: \tau$.

              If $\vdash e:\tau$, by the inversion lemma for \texttt{cat}, we have $\vdash e_1:\tau_1$ and $x:\tau_1 \vdash e_2:\tau$. Then by the substitution lemma, we have $\vdash [e_1/x]e_2: \tau$.
        \item Case $\inferrule*[Right=DLetL]{e_1\longrightarrow e_1'}{\mathtt{let}(e_1;x.e_2) \longrightarrow \mathtt{let}(e_1';x.e_2)}$

              By pattern match we know that $e= \mathtt{let}(e_1;x.e_2), e' = \mathtt{let}(e_1';x.e_2)$.

              WTS: If $\vdash e:\tau$, then $\mathtt{let}(e_1';x.e_2): \tau$.
              We use IH on the premise:
              \paragraph*{IH.} If $\vdash e_1:\tau_1$, then $\vdash e_1':\tau_1$.

              If $\vdash e:\tau$, then by the inversion lemma, we have $\vdash e_1:\tau_1$ and $x:\tau_1 \vdash e_2:\tau$. Then by IH we have $\vdash e_1':\tau_1$.
              Now by the substitution lemma, we have $\vdash [e_1'/e_1]e:\tau$. By the definition of substitution, we have $\vdash e': \tau$.
    \end{itemize}
\end{proof}

\subsubsection*{Inversion lemma}
\begin{lemma}
    (\texttt{cat}) If $\Gamma\vdash \mathtt{cat}(e_1,e_2):\mathtt{str}$, then $\Gamma \vdash e_1:\mathtt{str}$ and $\Gamma\vdash e_2:\mathtt{str}$.
\end{lemma}

\begin{proof}
    By induction on the derivation of $\Gamma\vdash \mathtt{cat}(e_1,e_2):\mathtt{str}$.
    \begin{itemize}
        \item Case $\inferrule*[Right=TCat]{\Gamma \vdash e_1:\mathtt{str} \\ \Gamma \vdash e_2:\mathtt{str}}{\Gamma \vdash \mathtt{cat}(e_1,e_2) : \mathtt{str}}$

              By the definition of \textsc{DCat}, we have $\Gamma\vdash e_1:\mathtt{str}$ and $\Gamma\vdash e_2:\mathtt{str}$.
    \end{itemize}
\end{proof}

\begin{lemma}
    (\texttt{let}) If $\Gamma\vdash \mathtt{let}(e_1;x.e_2):\tau$, then $\Gamma\vdash e_1:\tau'$ and $\Gamma,x:\tau'\vdash e_2:\tau$ for some $\tau'$.
\end{lemma}

\begin{proof}
    By induction on the derivation of $\Gamma\vdash \mathtt{let}(e_1;x.e_2):\tau$.
    \begin{itemize}
        \item Case $\inferrule*[Right=TLet]{\Gamma \vdash e_1:\tau_1 \\ \Gamma,x:\tau_1 \vdash e_2:\tau_2}{\Gamma \vdash \mathtt{let}(e_1;x.e_2) : \tau_2}$

              By the definition of \textsc{TLet}, we have $\Gamma\vdash e_1:\tau_1$ and $\Gamma,x:\tau_1 \vdash e_2:\tau_2$.
    \end{itemize}
\end{proof}

\begin{thebibliography}{9}
    \bibitem{int} Y. Yue. J. YuChen, “Proving Metatheorems and Type Safety.” Mar. 12, 2023.
\end{thebibliography}

\end{document}
