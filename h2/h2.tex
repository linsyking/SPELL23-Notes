\documentclass{article}
%include polycode.fmt
\usepackage[a4paper,left=2.5cm,right=2.5cm,top=\dimexpr15mm+1.5\baselineskip,bottom=2cm]{geometry}
\usepackage{amsmath}
\usepackage{graphicx}
\usepackage{setspace}
\usepackage{amsthm}
\usepackage{amssymb}
\usepackage{enumerate}
\usepackage{titlesec}
\usepackage{hyperref}
\usepackage{exscale}
\usepackage{pdfpages}
\usepackage{mathpartir}
\usepackage{relsize}
\usepackage{cases}
\usepackage{mathrsfs}
\usepackage{dutchcal}
\usepackage{float}
\usepackage{tcolorbox}
\usepackage{array}
\usepackage{xcolor}
\usepackage{stmaryrd}
\usepackage{calrsfs}
\usepackage{minted}
\DeclareMathAlphabet{\pazocal}{OMS}{zplm}{m}{n}

\theoremstyle{definition}
% \newtheorem{theorem}{Theorem}[section]
\newtheorem*{remark}{Remark}
% \newtheorem{corollary}{Corollary}[theorem]

% \SetWatermarkText{\mdfivesum file {\jobname}}
% \SetWatermarkFontSize{1cm}
\allowdisplaybreaks
\graphicspath{ {./figures/} }

% Change here
\newcommand{\course}{SPELL}
\newcommand{\asnum}{2}
\newcommand{\true}{\,\textsf{true}}
\newcommand{\brs}[1]{\llbracket#1\rrbracket_\sigma}
\newcommand{\lb}{$\lambda$}

\setcounter{section}{2}
\hypersetup{
    colorlinks, linkcolor=black, urlcolor=cyan
}
\title{\vspace{-3em}\course\, Lecture Notes \asnum\footnote{This note is licensed under a \href{https://creativecommons.org/licenses/by-nc-sa/2.0/}{CC BY-NC-SA 2.0} license.}}
\author{Y. Xiang\vspace{1em}}
\date{\today\vspace{-1em}}
\begin{document}
\maketitle
% \begin{spacing}{2}
\subsection{Lambda Calculus}

I recommend viewing the complete code at the \href{https://linsyking.github.io/lambda-playground/lambda.html?source=gist:93613afa70a3065aa03a989e5e7b537d}{playground}.
It is a fork from \href{https://github.com/wintercn/playground}{wintercn/playground}, which implements a JS to Lambda compiler.

\subsubsection*{Multiplication and Exponentiation}
We first write down the definition of church numerals and addition.

\begin{tcolorbox}
    \textsf{define} zero = \lb f.\lb z.z

    \textsf{define} one = \lb f.\lb z.f z

    \textsf{define} add = \lb x.\lb y.\lb f.\lb z.y f (x f z)

    \textsf{define} succ = \lb x.add x one
\end{tcolorbox}

We interpret $x\times y$ by ``adding $x$ $y$ times''. So it is natural to write out:

\begin{tcolorbox}
    \textsf{define} mul = \lb x.\lb y.y (add x) zero
\end{tcolorbox}

Similarly,

\begin{tcolorbox}
    \textsf{define} exp = \lb x.\lb y.y (mul x) one
\end{tcolorbox}

If x = y, then mul = \lb x.x (add x) zero, exp = \lb x.x (mul x) one.

\subsubsection*{Predecessor}

We can use the \textsf{pair} encoding.

If we want to get \textsf{pred}(x), then we can apply $(x, y) \rightarrow (y, y+1)$ to $(0, 0)$ x times, we get $(x-1,x)$.
Finally, we extract the first element.

Implementation:

\begin{tcolorbox}
    \textsf{define} TRUE = \lb x.\lb y.x

    \textsf{define} FALSE = \lb x.\lb y.y

    \textsf{define} pair = \lb x.\lb y.\lb c.c x y

    \textsf{define} first = \lb p.p TRUE

    \textsf{define} second = \lb p.p FALSE

    \textsf{define} pred = \lb x.first (x (\lb p.pair (second p) (succ (second p))) (pair zero zero))
\end{tcolorbox}

\subsubsection*{IFZ}

\begin{tcolorbox}
    \textsf{define} isZero = \lb x.x (\lb y.FALSE) TRUE

    \textsf{define} ifz = \lb n.\lb m0.\lb m1.(isZero n) m0 (m1 (pred n))
\end{tcolorbox}

\subsubsection*{$\alpha$-equivalence}

\begin{itemize}
    \item u and u. They are the same variables.
          Therefore by rule \textsc{refl} this is an $\alpha$-equivalence.
    \item u and v. They are all free variables, thus cannot be renamed. This is not a $\alpha$-equivalence.
    \item \lb z.z and \lb z.(\lb x.x) z. This is not a $\alpha$-equivalence, it requires $\beta$-reduction.
    \item \lb u.u x and \lb v.v x. Here u and v are \emph{bound} variables while x is free. By applying $\alpha-\lambda$ rule we can deduce this $\alpha$-equivalence.
    \item \lb u.u x and \lb v.v v. The latter is equal to \lb u.u u.
          The former term has a free variable x while the latter hasn't.
          Hence this is not an $\alpha$-equivalence.
\end{itemize}

\begin{remark}
    Two terms are alpha-equivalent iff one can be converted into the other purely by renaming \textbf{bound variables}.
\end{remark}

\subsubsection*{Locally Nameless Terms}

Let M = \lb x.(\lb y.y x) z = \lb x.z x, N = \lb x.\lb z.x y.

M(N) = z N = z (\lb x.\lb a.x y).

N(M) = \lb a.M y = \lb a.z y.

To represent M and N in the \emph{locally nameless terms} (LNT),
\begin{tcolorbox}
    M = \lb (\lb \fbox{0} \fbox{1}) z = \lb z \fbox{0}

    N = \lb \lb \fbox{1} y

    M(N) = z (\lb \lb \fbox{1} y)

    N(M) = \lb \lb \fbox{1} y (\lb z \fbox{0})
\end{tcolorbox}

Consider a function $\mathtt{substh}_n(\pazocal{M};\pazocal{N})$, it is recursively defined:
\begin{align*}
    \mathtt{substh}_n(x;\pazocal{N})                            & = x                                                                                            \\
    \mathtt{substh}_n(\fbox{$x$};\pazocal{N})                   & = \begin{cases}
                                                                        \pazocal{N},\quad x = n  \\
                                                                        \fbox{$x-1$},\quad x > n \\
                                                                        \fbox{$x$},\quad x < n
                                                                    \end{cases}                                                                     \\
    \mathtt{substh}_n(\lambda \pazocal{M};\pazocal{N})          & = \mathtt{substh}_{n+1}(\pazocal{M};\pazocal{N})                                               \\
    \mathtt{substh}_n(\pazocal{M}_1(\pazocal{M}_2);\pazocal{N}) & = (\mathtt{substh}_n(\pazocal{M}_1;\pazocal{N}))(\mathtt{substh}_n(\pazocal{M}_2;\pazocal{N}))
\end{align*}

Now, to substitute $\lambda \pazocal{M}$ with $\pazocal{N}$, we define \texttt{subst} function:
\begin{align*}
    \mathtt{subst}(\lambda\pazocal{M};\pazocal{N}) = \mathtt{substh}_0(\pazocal{M};\pazocal{N})
\end{align*}

I also implement the function \textsf{beta} in \textsf{LNT.hs}.

\subsubsection*{Conversion between LNT and \lb-terms}

See the code in the appendix.

After we define $\mathtt{bind}_n$, we can define function \texttt{n2l} recursively:
\begin{align*}
    \mathtt{n2l}(x)           & = x                                            \\
    \mathtt{n2l}(x\, y)       & = (\mathtt{n2l}(x))(\mathtt{n2l}(y))           \\
    \mathtt{n2l}(\lambda x.y) & = \lambda (\mathtt{bind}_0(x;\mathtt{n2l}(y)))
\end{align*}

To convert LNT back to \lb-terms, we have to first carefully choose the variable name to avoid conflicting with existing free variables.

Let $Var_\pazocal{M}$ be the set of the names of the free variables of $\pazocal{M}$.
Let $FR_\pazocal{M}(n)$ be an injective mapping that maps from an integer to a variable name (String) and $FR_\pazocal{M}(n)\notin Var_\pazocal{M}$.

Then we can implement a helper function $\mathtt{l2nh}$.
It is defined as:
\begin{align*}
    \mathtt{l2nh}_n(\pazocal{M};x)                            & = x                                                                                        \\
    \mathtt{l2nh}_n(\pazocal{M};\fbox{$x$})                   & = FR_\pazocal{M}(n - x -1)                                                                 \\
    \mathtt{l2nh}_n(\pazocal{M};\pazocal{M}_1(\pazocal{M}_2)) & = (\mathtt{l2nh}_n(\pazocal{M};\pazocal{M}_1))(\mathtt{l2nh}_n(\pazocal{M};\pazocal{M}_2)) \\
    \mathtt{l2nh}_n(\pazocal{M};\lambda \pazocal{N})          & = \lambda FR_\pazocal{M}(n).\mathtt{l2nh}_{n+1}(\pazocal{M};\pazocal{N})                   \\
\end{align*}

Now the $\mathtt{l2n}$ function can be defined:
\begin{align*}
    \mathtt{l2n}(\pazocal{M}) = \mathtt{l2nh}_0(\pazocal{M};\pazocal{M})
\end{align*}

\subsection*{Appendix}

Haskell code (read code on \href{https://github.com/linsyking/SPELL23-Notes/tree/master/h2/lnt}{github repo}):

\paragraph*{LNT.hs}
\inputminted{haskell}{lnt/src/LNT.hs}

\paragraph*{Normal.hs}
\inputminted{haskell}{lnt/src/Normal.hs}

\paragraph*{Translate.hs}
\inputminted{haskell}{lnt/src/Translate.hs}

\begin{thebibliography}{9}
    \bibitem{int} Y. Yue and J. Yuchen, “The Lambda Calculus.” Mar. 05, 2023.
\end{thebibliography}

\end{document}
