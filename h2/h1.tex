\documentclass{article}
\usepackage[a4paper,left=2.5cm,right=2.5cm,top=\dimexpr15mm+1.5\baselineskip,bottom=2cm]{geometry}
\usepackage{amsmath}
\usepackage{graphicx}
\usepackage{setspace}
\usepackage{amsthm}
\usepackage{amssymb}
\usepackage{enumerate}
\usepackage{titlesec}
\usepackage{hyperref}
\usepackage{exscale}
\usepackage{pdfpages}
\usepackage{mathpartir}
\usepackage{relsize}
\usepackage{cases}
\usepackage{mathrsfs}
\usepackage{dutchcal}
\usepackage{float}
\usepackage{tcolorbox}
\usepackage{array}
\usepackage{xcolor}
\usepackage{stmaryrd}

\theoremstyle{definition}
% \newtheorem{theorem}{Theorem}[section]
\newtheorem*{remark}{Remark}
% \newtheorem{corollary}{Corollary}[theorem]

% \SetWatermarkText{\mdfivesum file {\jobname}}
% \SetWatermarkFontSize{1cm}
\allowdisplaybreaks
\graphicspath{ {./figures/} }

% Change here
\newcommand{\course}{SPELL}
\newcommand{\asnum}{2}
\newcommand{\true}{\,\textsf{true}}
\newcommand{\brs}[1]{\llbracket#1\rrbracket_\sigma}
%

\setcounter{section}{2}
\hypersetup{
    colorlinks, linkcolor=black, urlcolor=cyan
}
\title{\vspace{-3em}\course\, Lecture Notes \asnum\footnote{This note is licensed under a \href{https://creativecommons.org/licenses/by-nc-sa/2.0/}{CC BY-NC-SA 2.0} license.}}
\author{Y. Xiang\vspace{1em}}
\date{\today\vspace{-1em}}
\begin{document}
\maketitle
% \begin{spacing}{2}
\subsection{Lambda Calculus}

\subsubsection*{Multiplication and Exponentiation}
We first write down the definition of church numerals and addition.

\begin{tcolorbox}
    \textsf{define} zero = $\lambda$f.$\lambda$z.z

    \textsf{define} one = $\lambda$f.$\lambda$z.f z

    \textsf{define} add = $\lambda$x.$\lambda$y.$\lambda$f.$\lambda$z.y f (x f z)

    \textsf{define} succ = $\lambda$x.add x one
\end{tcolorbox}

We interpret $x\times y$ by ``adding $x$ $y$ times''. So it is natural to write out:

\begin{tcolorbox}
    \textsf{define} mul = $\lambda$x.$\lambda$y.y (add x) zero
\end{tcolorbox}

Similarly,

\begin{tcolorbox}
    \textsf{define} exp = $\lambda$x.$\lambda$y.y (mul x) one
\end{tcolorbox}

If x = y, then mul = $\lambda$x.x (add x) zero, exp = $\lambda$x.x (mul x) one.

\subsubsection*{Predecessor}

We can use the \textsf{pair} encoding.

If we want to get \textsf{pred}(x), then we can apply $(x, y) \rightarrow (y, y+1)$ to $(0, 0)$ x times, we get $(x-1,x)$.
Finally, we extract the first element.

Implementation:

\begin{tcolorbox}
    \textsf{define} TRUE = $\lambda$x.$\lambda$y.x

    \textsf{define} FALSE = $\lambda$x.$\lambda$y.y

    \textsf{define} pair = $\lambda$x.$\lambda$y.$\lambda$c.c x y

    \textsf{define} first = $\lambda$p.p TRUE

    \textsf{define} second = $\lambda$p.p FALSE

    \textsf{define} pred = $\lambda$x.first (x ($\lambda$p.pair (second p) (succ (second p))) (pair zero zero))
\end{tcolorbox}

\subsubsection*{IFZ}

\begin{tcolorbox}
    \textsf{define} ifz = $\lambda$n.$\lambda$m0.$\lambda$m1.IF (isZero n) m0 (m1 (pred n))
\end{tcolorbox}

\subsubsection*{$\alpha$-equivalence}

\begin{itemize}
    \item u and u. They are the same variables.
          Therefore by rule \textsc{refl} this is an $\alpha$-equivalence.
    \item u and v. They are all free variables, thus cannot be renamed. This is not a $\alpha$-equivalence.
    \item $\lambda$z.z and $\lambda$z.($\lambda$x.x) z. This is not a $\alpha$-equivalence, it requires $\beta$-reduction.
    \item $\lambda$u.u x and $\lambda$v.v x. Here u and v are \emph{bound} variables while x is free. By applying $\alpha-\lambda$ rule we can deduce this $\alpha$-equivalence.
\end{itemize}

\begin{remark}
    Two terms are alpha-equivalent iff one can be converted into the other purely by renaming \textbf{bound variables}.
\end{remark}

\subsection*{Appendix}
See \href{https://linsyking.github.io/lambda-playground/lambda.html?source=gist:93613afa70a3065aa03a989e5e7b537d}{JS to Lambda}.

\begin{thebibliography}{9}
    \bibitem{int} Y. Yue and J. Yuchen, “The Lambda Calculus.” Mar. 05, 2023.
\end{thebibliography}

\end{document}
